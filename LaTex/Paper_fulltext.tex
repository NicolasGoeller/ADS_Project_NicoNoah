%% 
%% Copyright 2007-2018 Elsevier Ltd
%% 

\documentclass[preprint,12pt,authoryear]{elsarticle}

\usepackage{blindtext}
%% For including figures, graphicx.sty has been loaded in
%% elsarticle.cls. If you prefer to use the old commands
%% please give \usepackage{epsfig}

%% The amssymb package provides various useful mathematical symbols
%%\usepackage{amssymb}
%% The amsthm package provides extended theorem environments
%% \usepackage{amsthm}

%% The lineno packages adds line numbers. Start line numbering with
%% \begin{linenumbers}, end it with \end{linenumbers}. Or switch it on
%% for the whole article with \linenumbers.
%% \usepackage{lineno}

\journal{Dr. Florian Bader}

\begin{document}

\begin{frontmatter}

%% Title, authors and addresses

%% use the tnoteref command within \title for footnotes;
%% use the tnotetext command for theassociated footnote;
%% use the fnref command within \author or \address for footnotes;
%% use the fntext command for theassociated footnote;
%% use the corref command within \author for corresponding author footnotes;
%% use the cortext command for theassociated footnote;
%% use the ead command for the email address,
%% and the form \ead[url] for the home page:
%% use optional labels to link authors explicitly to addresses:
%% \author[label1,label2]{}
%% \address[label1]{}
%% \address[label2]{}

\title{Visualising Life Satisfaction in Countries across Europe: Individual- and Multi-Level Evidence on Cultural and Pecuniary Factors}

\author{Nicolas F. K. Göller\fnref{label2}}
\ead{n.goeller@zeppelin-university.net}
\fntext[label2]{Student in Sociology, Politics and Economics}

\author{Noah Peters\fnref{label1}}
\ead{no.peters@zeppelin-university.net}
\fntext[label2]{Student in Sociology, Politics and Economics}

\address{Zeppelin University, Fallenbrunnen 3, 88045 Friedrichshafen\fnref{label3}}
%% \fntext[label3]{}


\begin{abstract}
%% Text of abstract
This project was set out to provide an easy access to a comprehensible way for exploring the differences in determinants of life satisfaction 
all over Europe. While the actual analysis and presentation of results will be shown in different ways, this document
shall provide an overview over the data used and the treatment of variables applied during the project. 
The project's aim is not only to inspire interest into the determinants of happiness in human life, but also to encourage
later researchers to critically enquire with the work done here in order to point out its shortcomings and improve the final
result. Thus this document should provide the necessary transparency about sourcing of data and work on variables that
has been done previously to conducting the analysis and visualising results.
\end{abstract}

\end{frontmatter}

%% main text
\newpage
\tableofcontents
\newpage


\section{Introduction and Preliminary remarks}
Anecdotal evidence suggests that people spend a significant share of their lifetime working (for a casual calculation 
see: \citet{thompson_what_2016}). Moreover, people's work environment is changing constantly and especially today's 
discussion about the impact of digitalisation on the labour market requires to investigate how occupational factors shape 
people's life in general. Since work constitutes a considerable part of people’s life, satisfaction with one’s job and the
occupational environment must severely determine overall well-being. Drawing on the above-mentioned diversity of 
occupations, their respective circumstances and adaptation to the ‘digital revolution’, this assumption gains even more 
ground and relevance: \textit{If work changes, life changes.} \\
But which work-related factors contribute to a happy - or unhappy - life? Also, how does life satisfaction of the employed
vary across regions and countries? \\
Using data form the European Values Survey (EVS), we calculated multiple regression models to assess the causality of
work-related factors and well-being on the individual level. Moreover, multi-level analyses with regard to social capital and
macroeconomic as well as developmental indicators were performed to account for variation across regions and countries. 

\section{Theory and Literature Review}
Although the causal relation between work and happiness/life satisfaction seems straightforward and intuitively correct,
its academic investigation is far from trivial. On the one hand, this appears in light of definitional questions regarding the 
wo variables of interest. On the other hand, operationalising the constructs appropriately remains subject to discussions
(for subjective well-being see: \cite{kahneman_developments_2006,layard_measuring_2010}). Lastly, the causal nature of job and life
satisfaction presents itself as far more complex than intuitively assumed. \\
Each of these three fields is addressed by a comprehensive literature review to set the tone for the subsequent analysis.
As this analysis draws on specific independent variables on the individual level and several factors on higher aggregate
levels (multi-level analysis), their respective relation to life satisfaction, as assessed in the literature, is discussed as well.

\subsection{Life Satisfaction as a Component of Subjective Well-Being (SWB)}
So far, the terms life satisfaction, happiness, and well-being have been used interchangeably. In fact, however, they
are commonly understood to act upon each other therefore constituting the wider concept of subjective well-being (SWB)
that includes life satisfaction, happiness, and positive affect  \citep{bowling_meta-analytic_2010,diener_subjective_1984}. 
Paying debt to the European Value Study’s (EVS) operationalisation of SWB, we only include life satisfaction as a
dependent variable: While life satisfaction is measured using a ten-point scale the other available component of SWB,
happiness, comes in a four-point scale. Consequently, we only included the life satisfaction variable for our regression
models. For the remains of this study, we concentrate on life satisfaction as a dependent variable. The variable selection
is further elaborated upon in the method section. 

\subsection{SWB and Job Satisfaction}
The literature examining the causal relation of SWB and job satisfaction is extensive
(for reviews see: \citet{bowling_meta-analytic_2010,erdogan_whistle_2012}). \citet{bowling_meta-analytic_2010}, in their
meta-analysis, show a positive relationship between all three respective components of SWB and job satisfaction.
Going even further, \citet{near_job_1987} postulate that “job satisfaction must logically constitute some part of the
broader construct of life satisfaction; thus, the question regarding spillover between the two must be ‘how much’ not
‘whether’." (p. 398). \\
Before we examine the how much-question, it is necessary to discuss the direction of causality between SWB and job
satisfaction. In general, the present literature differentiates two approaches: the part-whole theory
\citep{bowling_meta-analytic_2010,hart_predicting_1999,judge_dispositional_1998} and the dispositional approach. The former
proposes job satisfaction to act upon SWB. This appears because of the simple reason that has already been mentioned
above: job satisfaction constitutes a sub-category of overall life satisfaction \citep[cf.]{near_job_1987}.
Consequently, the approach is also conceived of as bottom-up approach \citep{erdogan_whistle_2012}. Conversely, the
dispositional approach assumes a general predisposition of satisfaction with life to affect more specific domains like work
i.e., satisfaction with those domains \citep{bowling_meta-analytic_2010,diener_subjective_1984,headey_top-down_1991,judge_job_1993,judge_another_1993}. 
Erdogan et al. (2012) link these predispositions to personal traits that determine how satisfied people feel with certain
domains and life in general (top-down approach). Positioned between the poles is the spillover hypothesis that just
contents the basic linkage between job and life satisfaction or even other domains of life 
\citep{bowling_meta-analytic_2010,erdogan_whistle_2012,judge_job_2001,judge_individual_1994}.
Conceptually, this relation can be understood as a reciprocal relation between the various domains of life and people’s
satisfaction (Bowling et al., 2010). \\
While the direction of causality between life satisfaction and job satisfaction is an open discussion, the interplay of life
satisfaction with other work-related variables might be easier to grasp. Indeed, our study also examines the effects of
variables that characterise the working environment beyond perceived job satisfaction. This wider scope appears
reasonable since “job satisfaction is only a partial indicator of satisfaction with work domain. A person’s satisfaction with
career and work-related stress are important elements of the work domain that also need to be understood.” \citep[p. 1042]{erdogan_whistle_2012}.


%% The Appendices part is started with the command \appendix; appendix sections are then done as normal sections
%% \appendix

%% \section{}
%% \label{}

%% If you have bibdatabase file and want bibtex to generate the bibitems, please use
\bibliographystyle{elsarticle-harv} 
\bibliography{Literature}


\end{document}

\endinput
%%

