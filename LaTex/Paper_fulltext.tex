%% 
%% Copyright 2007-2018 Elsevier Ltd
%% 

\documentclass[preprint,12pt,authoryear]{elsarticle}

%% For including figures, graphicx.sty has been loaded in
%% elsarticle.cls. If you prefer to use the old commands
%% please give \usepackage{epsfig}

%% The amssymb package provides various useful mathematical symbols
%%\usepackage{amssymb}
%% The amsthm package provides extended theorem environments
%% \usepackage{amsthm}
\usepackage{url}
\usepackage{natbib}
%% The lineno packages adds line numbers. Start line numbering with
%% \begin{linenumbers}, end it with \end{linenumbers}. Or switch it on
%% for the whole article with \linenumbers.
%% \usepackage{lineno}

\journal{Dr. Florian Bader}

\begin{document}
	
	\begin{frontmatter}
		
		%% Title, authors and addresses
		
		%% use the tnoteref command within \title for footnotes;
		%% use the tnotetext command for theassociated footnote;
		%% use the fnref command within \author or \address for footnotes;
		%% use the fntext command for theassociated footnote;
		%% use the corref command within \author for corresponding author footnotes;
		%% use the cortext command for theassociated footnote;
		%% use the ead command for the email address,
		%% and the form \ead[url] for the home page:
		%% use optional labels to link authors explicitly to addresses:
		%% \author[label1,label2]{}
		%% \address[label1]{}
		%% \address[label2]{}
		
		\title{Visualising Life Satisfaction in Countries across Europe: Individual- and Multi-Level Evidence on Cultural and Pecuniary Factors}
		
		\author{Nicolas F. K. Göller\fnref{label2}}
		\ead{n.goeller@zeppelin-university.net}
		\fntext[label2]{Student in Sociology, Politics and Economics}
		
		\author{Noah Peters\fnref{label1}}
		\ead{no.peters@zeppelin-university.net}
		\fntext[label2]{Student in Sociology, Politics and Economics}
		
		\address{Zeppelin University, Fallenbrunnen 3, 88045 Friedrichshafen\fnref{label3}}
		%% \fntext[label3]{}
		
		
		\begin{abstract}
			%% Text of abstract
			This project was set out to provide an easy access to a comprehensible way for exploring the differences in determinants of life satisfaction 
			all over Europe. While the actual analysis and presentation of results will be shown in different ways, this document
			shall provide an overview over the data used and the treatment of variables applied during the project. 
			The project's aim is not only to inspire interest into the determinants of happiness in human life, but also to encourage
			later researchers to critically enquire with the work done here in order to point out its shortcomings and improve the final
			result. Thus this document should provide the necessary transparency about sourcing of data and work on variables that
			has been done previously to conducting the analysis and visualising results.
		\end{abstract}
		
	\end{frontmatter}
	
	%% main text
	\newpage
	\tableofcontents
	\newpage
	
	
	\section{Introduction and Preliminary remarks}
	Anecdotal evidence suggests that people spend a significant share of their lifetime working (for a casual calculation 
	see: \citet{thompson_what_2016}). Moreover, people's work environment is changing constantly and especially today's 
	discussion about the impact of digitalisation on the labour market requires to investigate how occupational factors shape 
	people's life in general. Since work constitutes a considerable part of people’s life, satisfaction with one’s job and the
	occupational environment must severely determine overall well-being. Drawing on the above-mentioned diversity of 
	occupations, their respective circumstances and adaptation to the ‘digital revolution’, this assumption gains even more 
	ground and relevance: \textit{If work changes, life changes.} \\
	But which work-related factors contribute to a happy - or unhappy - life? Also, how does life satisfaction of the employed
	vary across regions and countries? \\
	Using data form the European Values Study (EVS) and several other sources, we calculated multiple regression models to assess the causality of
	work-related factors and well-being on the individual level. Moreover, multi-level analyses with regard to social capital and
	macroeconomic as well as developmental indicators were performed to account for variation across regions and countries. 
	%% Könntest du den Fokus ein wenig weg von Work und hin auf multilevel + pecuniary/ non-pecuiary setzen, für so viel Fokus haben wir nicht genug work-relatedness
	\section{Motivational Manifesto}
	In a time of widespread digitalisation for many aspects of modern society, 'datafication' and subsequent analysis of data has become ever more prevalent \citep{baack_datafication_2015, lycett_datafication:_2013}. Transparency of data sourcing, data access and methods employed in the course of this analysis is a chief concern of the 'Open Science' movement \citep{delfanti_open_2010}. With more and more transparency, reproducibility of procedures is expected to increase and thus the quality of insights by replication and and improvements of methods through a 'wisdom of the (educated) crowd'. There have been extensive developments in the form of guidelines and protocols for researchers, universities and scientific journals that are aimed at promoting more transparency in the scientific process \citep{nosek_promoting_2015, miguel_promoting_2014}. 
	At the same time, large shares of the population lack the essential skills to participate in the research process or perceive the barrier of entry in acquiring the necessary skills to do so, as insurmountably high. The concept of 'Citizen Science' has thus emerged sitting at the edge of both science and education. Citizen science projects enlist non-scientist members of the public to participate in the research (mainly in data collection), allowing to survey large amounts of data, while educating participants about the research topic \citep{shirk_public_2012}. Originally coming from environmental research this approach has already allowed researcher to survey wide ranges of locations and time periods and thus provides access to high-quality data that could quite impossibly be attained by professional research teams on their own \citep{bonney_citizen_2009}.
	
	Conceptually, 'Open Science' and 'Citizen Science' are deeply intertwined as 'Citizen Science' needs to be as transparent and easily accessible as possible to be effective on a large scale \citep{nov_dusting_2011} and 'Open Science' needs to offer educational value and low entry barriers if it wants to gain attention and participation of a broader audience \citep{newman_future_2012}. We feel committed to the mission of these movements and have thus designed our project in order to advance it for the social sciences. 
	
	\section{Theory and Literature Review}
	Although the causal relation between work and happiness/life satisfaction seems straightforward and intuitively correct,
	its academic investigation is far from trivial. On the one hand, this appears in light of definitional questions regarding the 
	wo variables of interest. On the other hand, operationalising the constructs appropriately remains subject to discussions
	(for subjective well-being see: \citet{kahneman_developments_2006,layard_measuring_2010}). Lastly, the causal nature of job and life
	satisfaction presents itself as far more complex than intuitively assumed. \\
	Each of these three fields is addressed by a comprehensive literature review to set the tone for the subsequent analysis.
	As this analysis draws on specific independent variables on the individual level and several factors on higher aggregate
	levels (multi-level analysis), their respective relation to life satisfaction, as assessed in the literature, is discussed as well.
	
	\subsection{Life Satisfaction as a Component of Subjective Well-Being (SWB)}
	So far, the terms life satisfaction, happiness, and well-being have been used interchangeably. In fact, however, they
	are commonly understood to act upon each other therefore constituting the wider concept of subjective well-being (SWB)
	that includes life satisfaction, happiness, and positive affect  \citep{bowling_meta-analytic_2010,diener_subjective_1984}. 
	Paying debt to the European Value Study’s (EVS) operationalisation of SWB, we only include life satisfaction as a
	dependent variable: While life satisfaction is measured using a ten-point scale the other available component of SWB,
	happiness, comes in a four-point scale. Consequently, we only included the life satisfaction variable for our regression
	models. For the remains of this study, we concentrate on life satisfaction as a dependent variable. The variable selection
	is further elaborated upon in the method section. 
	
	\subsection{SWB and Job Satisfaction}
	The literature examining the causal relation of SWB and job satisfaction is extensive
	(for reviews see: \citet{bowling_meta-analytic_2010,erdogan_whistle_2012}). \citet{bowling_meta-analytic_2010}, in their
	meta-analysis, show a positive relationship between all three respective components of SWB and job satisfaction.
	Going even further, \citet{near_job_1987} postulate that “job satisfaction must logically constitute some part of the
	broader construct of life satisfaction; thus, the question regarding spillover between the two must be ‘how much’ not
	‘whether’." (p. 398). \\
	Before we examine the how much-question, it is necessary to discuss the direction of causality between SWB and job
	satisfaction. In general, the present literature differentiates two approaches: the part-whole theory
	\citep{bowling_meta-analytic_2010,hart_predicting_1999,judge_dispositional_1998} and the dispositional approach. The former
	proposes job satisfaction to act upon SWB. This appears because of the simple reason that has already been mentioned
	above: job satisfaction constitutes a sub-category of overall life satisfaction \citep[cf.]{near_job_1987}.
	Consequently, the approach is also conceived of as bottom-up approach \citep{erdogan_whistle_2012}. Conversely, the
	dispositional approach assumes a general predisposition of satisfaction with life to affect more specific domains like work
	i.e., satisfaction with those domains \citep{bowling_meta-analytic_2010,diener_subjective_1984,headey_top-down_1991,judge_job_1993,judge_another_1993}. 
	Erdogan et al. (2012) link these predispositions to personal traits that determine how satisfied people feel with certain
	domains and life in general (top-down approach). Positioned between the poles is the spillover hypothesis that just
	contents the basic linkage between job and life satisfaction or even other domains of life 
	\citep{bowling_meta-analytic_2010,erdogan_whistle_2012,judge_job_2001,judge_individual_1994}.
	Conceptually, this relation can be understood as a reciprocal relation between the various domains of life and people’s
	satisfaction (Bowling et al., 2010). \\
	While the direction of causality between life satisfaction and job satisfaction is an open discussion, the interplay of life
	satisfaction with other work-related variables might be easier to grasp. Indeed, our study also examines the effects of
	variables that characterise the working environment beyond perceived job satisfaction. This wider scope appears
	reasonable since “job satisfaction is only a partial indicator of satisfaction with work domain. A person’s satisfaction with
	career and work-related stress are important elements of the work domain that also need to be understood.” \citep[p. 1042]{erdogan_whistle_2012}.
	
	\subsection{The Wider Interplay of Work-Related Variables with Life Satisfaction}
Our analysis features a myriad of variables that can be examined with regard to their effect on overall life satisfaction.
In order to justify our variable selection the following remarks elucidate the respective variables' interplay with life satisfaction.
In contrast to the previous examination of the overall concept of SWB, to accounts on the different variables are presented 
in a briefer manner. Rather than covering the entire theoretical background, the following elaboration is though to spur interest
and curiosity to examine the relation of life satisfaction and other factors on one's own - aided by our interactive 
European Happiness Observer interface. 

\subsubsection{Life satisfaction and income}
The life satisfaction-income discussion almost always refers back to \citet{easterlin_does_1974} whose observations regarding
the relation of life satisfaction i.e., happiness and income were dubbed the `Easterlin Paradox'. Indeed, his findings from time series data
suggest that life satisfaction remains flat, despite increases in GDP per capita. These findings still heavily inform other researchers'
work and indicate that the relative income in comparision to others might influence SWB \citep{di_tella_happiness_2010,drichoutis_reference_2010,pedersen_happiness_2011}.
Emphasising relative position follows the lead of behavioural labour economics that builds on the findings of prospect theory \citep{kahneman_prospect_1979}.
This concept implies that reference points rather than absolute positions inform people's choices. In the course of behavioural
labour economics, this backgroundis highly relevant \citep{dohmen_behavioral_2014}. 
While investigations examining relative status and changes over time appear highly relevant and theoretically justified, the present
project ist not suited to cover these questions. Not being in posession of either time series or relative income data, we need
to focus on absolute income at a certain point in time. Although this independent variable might not capture the underlying relation
of income and life satisfaction, it qualifies as a basic means of inquiry that sheds light on the distinction between life satisfaction
and happiness: “high income buys life satisfaction but not happiness.” \citep[p. 16489]{kahneman_high_2010}. 

\subsubsection{Pecuniary vs. non-pecuniary factors}
Income-related investigations also stimulate inquiries examining the opposite of monetary factors, namely non-pecuniary
variables. \textit{Since high income might not be all to strive for, which `softer' factors determine a satisfactory life?}
In general, health takes centre-stage as a determinant of life satisfaction. This statement is not only backed by common 
sense or daily wisdom (Who has not said or been told that health is life's most precious gift?) but also by empirical observations \citep{pedersen_happiness_2011}. \\
In the working environment, health is not a trivial factor either. The \citet{who_global_1994} stresses the importance of health at the work place and terms like `work-life-balance' and `Generation Y' imply changing perceptions of a pleasant working environment. Also \citet{andersson_happiness_2008}
investigates the special relation between self-employment and life satisfaction with ample regard to health issues. \\
Self-employment as such exhibits a variable that is thought to affect life satisfaction \citep{andersson_happiness_2008,schneck_why_2014,van_der_zwan_self-employment_2018}
due to non-pecuniary circumstances that are innate and to the status of self-employment. \\
This remark also leads to a greater discussion of features proposed by self-determinantion theory \citep{deci_intrinsic_1985,deci_self-determination_1989,ryan_self-determination_2000}.
Lastly, \citet{luo_essays_2018} uses EVS data estimate the effects of pecuniary and non-pecuniary factors on life satisfaction
The EVS, for example, captures to which degree respondents feel the need for a job that enables employees to develop their talents. \\
of the unemployed. Intuitively, financial uncertainty might appear as the major concern of unemployment but further investigations
show the significant role that social and psychological factors play. This observation can be interpreted as a general plea for research on
the interplay of non-pecuniary factors of the work environment and life satisfaction. 

\subsubsection{Non-pecuniary factors of the work environment}
Drawing on the previous remarks regarding in incorporation of work-related non-pecuniary variables, our analysis comprises of
multiple factors that touch on leadership responsibilities, self-determination related factors, or the occupation's reputation.
The following variables can be examined using the European Happiness Observer:
\begin{itemize}
   \item Work importance in life
   \item Freedem of job decisions
   \item Job development
   \item Work duty to society
   \item Work should come always first
   \item Supervising responsibilities 
   \item Status of employment (i.e., being self-employed)
   \item International Socio-Economic Index of Occupational Status   
\end{itemize}

\subsection{Macro-Level Variables and Multilevel Analyses}
So far, the theoretical background sketched relations between life satisfaction and various individual-level variables. While
this perspective enables us to assess psychological constructs and individual life realities in general, a core feature of our
analysis is the incorporation of multilevel investigations. \\
Being interested in life satisfaction across European countries and regions, we examined variables that might account for 
between-variance. The background on these factors and their respective relation to life satisfaction is examined in this 
section. We point out that the macro-variables cover a wider array than the work-related aspects described above. Nonetheless,
the macro-variables and the multilevel models can be used to supplement individual-level analyses as it is common practice
in multilevel contexts.

\subsubsection{Social capital} 
\citet{bartolini_happy_2014} inspect the relation between life satisfaction and social capital on a micro-level and resume that
social capital can explain variation in life satisfaction in the medium- and long-run while the association is less striking in a
short-run perspective. \citet{mikucka_when_2017} underpin this verdict with their multilevel analysis using World Values Study 
and EVS data. While we did not obtain time-series data to inspect associations over time, we recommend to incorporate 
social capital as a regional level-two variable (here: European NUTS1 regions). Referring to Putnam, \citet{Savelkoul_explaining_2011} imply that social 
capital is especially related to the regional level. \\
Having set the level of inspection, we focussed on the operationalisation of social capital given EVS data. \citet{van_schaik_social_2002}
recommends a procedure specifically related to the EVS that builds on using several categories of social capital as a proxy
for the overall concept. In this fashion, we constructed two indices for (i) institutional trust and (ii) trustworthiness that measures
the prevalence of social norms legal compliance. Additionally, the EVS includes a binary item covering interpersonal trust. \citet{van_schaik_social_2002}
also associates the respective categories of social capital with European NUTS1 regions, as we do. \\
Comprehensive plots of the proxies can be found in our Observer app and multilevel regressions can be computed. 

\subsubsection{Economic macro-data}
Building on \citet{easterlin_does_1974} and the diverse remarks on the association of life satisfaction (see above), GDP presents itself
as a useful multilevel factor. We use country data on GDP for 2008 (the year of the EVS's fourth wave) as a level-3 variable.
Furthermore, Human Development Index (HDI) data can be incorporated to capture a wider socio-economic context. 

\subsubsection{Freedom House Ratings}
In order to inspect level-3 effects regarding differences in the political system, we offer to incorporate various Freedom House 
indicators measuring a country's level of democracy. While this perspective offers an additional flavour, it lies not within the main
focus of this project. \\
A detailed description of all data used for this project is presented in the following section. 	
	
\section{Data}
All data employed in this project is acquired from official sources and with permission of the provider or in the course of open access. Data as well as original codebooks will be made available for the user upon request, or delivered within the project.
The project is mainly built on data from the 'European Value Study 2008' (EVS), which is the fourth survey wave \citep{evs_european_2016}. The EVS contains questions regarding values of Europeans on aspects like moral, work, family, politics, society, religion, as well as an extensive collection of socio-demographic items. Issues and questions regarding surveying of the data, errata, harmonizing of results etc. may be settled with the publisher or from the original codebook. The file contains a total 66281 observations for 477 variables over 46 European countries. All transformations on variables can be read in detail from the codebook. Significant changes to the data were made through exclusion of 'Russian Federation' from the sample for the purpose of better visualisation. Furthermore, 'Northern Cyprus' and 'Northern Ireland' which were coded as own countries were added to their respective states after international law. Thus, data for 43 countries was processed for use in the project. Furthermore, aggregations were made from the data in total of 5 cases to create variables that are applicable for multilevel analysis. Finally, a variable of 'Geographical region' was created from the country variable in the EVS. This classification was based on the 'Standard Country or Area Codes for Statistics Use' from the UN Statistics Division \citep{unsd_standard_2019}.
Structural data for nations was obtained from the World Bank for 'GDP per capita of 2008' (in 2018 dollars per inhabitant)\citep{world_bank_gdp_2019}, and for 'Gini Coefficient 2008' (in ration scale 0-1)\citep{world_bank_gini_2019}. In the case of 'Gini Coefficient 2008', missing values for Germany, Poland, Serbia, Kosovo, Macedonia, Croatia and Bosnia Herzegovina were taken from Eurostat 2008 (Germany, Poland) \citep{eurostat_eurostat_2019}, from the CIA World Factbook (Serbia 2008) \citep{central_intelligence_agency_serbia_2019}, from World Bank Gini Coefficient 2009 (Kosovo, Macedonia, Croatia), and from World Bank Gini Coefficient 2007 (Bosnia Herzegovina) \citep{world_bank_gini_2019}. Data was also obtained from the 'Democracy Cross-National' dataset created by Pippa Norris \citep{norris_data_2015} for the Freedom House Democracy rating 2008 and the Human Development Index 2008. Missing values of the Freedom House rating for Kosovo \citep{freedom_house_kosovo_2012} and Montenegro \citep{freedom_house_montenegro_2012} were replaced by the indication for 2008 on the Freedom House website. The missing value of the HDI 2008 for Kosovo was approximated by the value for Serbia in 2008.
	
\section{User Manual}
	
\section{Conclusion}

	
	%% The Appendices part is started with the command \appendix; appendix sections are then done as normal sections
\newpage
 \appendix
	
\section{Responsibility for Project Areas}

\begin{tabular}{|c|c|c|}
	\hline 
	\textbf{Project area}	& \textbf{Noah Peters} 	& \textbf{Nicolas Goeller} 	\\ 
	\hline 
	Data wrangling			&  		   				&  				 			\\ 
	\hline 
	Variable preparation 	&  		   				&  				 			\\ 
	\hline 
	Geodata					&  		   				&  				 			\\ 
	\hline 
	Git	     			 	&  		   				&  				 			\\ 
	\hline 			
	Indices				 	&  		   				&  				 			\\ 
	\hline 
	Paper visualisations 	&  		   				&  				 			\\ 
	\hline 
	Shiny 				 	&  		   				&  				 			\\ 
	\hline 
	Codebook 			 	&  		   				&  				 			\\ 
	\hline 
	Geoplots 			 	&  		   				&  				 			\\ 
	\hline 
	LaTex				 	&  		   				&  				 			\\ 
	\hline 
	Motivation			 	&  		   				&  				 			\\ 
	\hline 
	Theory				 	&  		   				&  				 			\\ 
	\hline 
	User manual			 	&  		   				&  				 			\\ 
	\hline 
\end{tabular}

\newpage
%% If you have bibdatabase file and want bibtex to generate the bibitems, please use
\bibliographystyle{elsarticle-harv} 
\bibliography{Literature}



\end{document}

\endinput
%%

