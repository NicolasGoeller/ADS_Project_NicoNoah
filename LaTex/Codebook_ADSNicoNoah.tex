%% 
%% Copyright 2007-2018 Elsevier Ltd
%% 

\documentclass[preprint,12pt,authoryear]{elsarticle}

\usepackage{blindtext}
%% For including figures, graphicx.sty has been loaded in
%% elsarticle.cls. If you prefer to use the old commands
%% please give \usepackage{epsfig}

%% The amssymb package provides various useful mathematical symbols
%%\usepackage{amssymb}
%% The amsthm package provides extended theorem environments
%% \usepackage{amsthm}

%% The lineno packages adds line numbers. Start line numbering with
%% \begin{linenumbers}, end it with \end{linenumbers}. Or switch it on
%% for the whole article with \linenumbers.
%% \usepackage{lineno}

\journal{Nuclear Physics B}

\begin{document}

\begin{frontmatter}

%% Title, authors and addresses

%% use the tnoteref command within \title for footnotes;
%% use the tnotetext command for theassociated footnote;
%% use the fnref command within \author or \address for footnotes;
%% use the fntext command for theassociated footnote;
%% use the corref command within \author for corresponding author footnotes;
%% use the cortext command for theassociated footnote;
%% use the ead command for the email address,
%% and the form \ead[url] for the home page:
%% \title{Title\tnoteref{label1}}
%% \tnotetext[label1]{}
%% \author{Name\corref{cor1}\fnref{label2}}
%% \ead{email address}
%% \ead[url]{home page}
%% \fntext[label2]{}
%% \cortext[cor1]{}
%% \address{Address\fnref{label3}}
%% \fntext[label3]{}

\title{Codebook for the seminar project 'European Happiness Observer' in the lecture 'Advanced Data Science with R'}

%% use optional labels to link authors explicitly to addresses:
%% \author[label1,label2]{}
%% \address[label1]{}
%% \address[label2]{}

\author{Noah Peters, Nicolas Göller}

\address{Zeppelin University, Fallenbrunnen 3, 88045 Friedrichshafen}

\end{frontmatter}

%% \linenumbers

%% main text
\tableofcontents
\newpage

\section{Introduction and preliminary remarks}
\label{}
This project was set out to provide an easy access to a comprehensible way for exploring the differences in determinants of life satisfaction all over Europe. While the actual analysis and presentation of results will be shown in different ways, this document shall provide an overview over the data used and the treatment of variables applied during the project. The project's aim is not only to inspire interest into the determinants of happiness in human life, but also to encourage later researchers to critically enquire with the work done here in order to point out its shortcomings and improve the final result. Thus this document should provide the necessary transparency about sourcing of data and work on variables that has been done previously to conducting the analysis and visualising results.
 
\section{Data access and documentation}
\label{}

\section{Description of data}
\label{}
\subsection{Countries observed}
\subsection{Period of data collection}
\subsection{Structure of the data file}

\section{Variable treatment}
\label{}
\subsection{Base variables}
Country code, Country of residence: c\_code, nation \\ \noindent\hspace*{10mm}%
	Variable class: character, factor \\ \noindent\hspace*{10mm}%
	Content: \\ \noindent\hspace*{20mm}%
		AL, Albania\\ \noindent\hspace*{20mm}%
		AM, Armenia\\ \noindent\hspace*{20mm}%
		AT, Austria\\ \noindent\hspace*{20mm}%
		BA, Bosnia Herzegovina\\ \noindent\hspace*{20mm}%
		BE, Belgium\\ \noindent\hspace*{20mm}%
		BG, Bulgaria; BY Belarus\\ \noindent\hspace*{20mm}%
		CH, Switzerland\\ \noindent\hspace*{20mm}%
		CY, Cyprus\\ \noindent\hspace*{20mm}%
		CZ, Czech Republic\\ \noindent\hspace*{20mm}%
		DE, Germany\\ \noindent\hspace*{20mm}%
		DK, Denmark\\ \noindent\hspace*{20mm}%
		EE, Estonia\\ \noindent\hspace*{20mm}%
		ES, Spain\\ \noindent\hspace*{20mm}%
		FI, Finland\\ \noindent\hspace*{20mm}%
		FR, France\\ \noindent\hspace*{20mm}%
		GE, Georgia\\ \noindent\hspace*{20mm}%
		GR, Greece\\ \noindent\hspace*{20mm}%
		HR, Croatia\\ \noindent\hspace*{20mm}%
		HU, Hungary\\ \noindent\hspace*{20mm}%
		IE, Ireland\\ \noindent\hspace*{20mm}%
		IS, Iceland\\ \noindent\hspace*{20mm}%
		IT, Italy\\ \noindent\hspace*{20mm}%
		RS-KM, Kosovo\\ \noindent\hspace*{20mm}%
		LT, Lithuania\\ \noindent\hspace*{20mm}%
		LU, Luxembourg\\ \noindent\hspace*{20mm}%
		LV, Latvia\\ \noindent\hspace*{20mm}%
		MD, Moldova\\ \noindent\hspace*{20mm}%
		ME, Montenegro\\ \noindent\hspace*{20mm}%
		MK, Macedonia\\ \noindent\hspace*{20mm}%
		MT, Malta\\ \noindent\hspace*{20mm}%
		NL, Netherlands\\ \noindent\hspace*{20mm}%
		NO, Norway\\ \noindent\hspace*{20mm}%
		PL, Poland\\ \noindent\hspace*{20mm}%
		PT, Portugal\\ \noindent\hspace*{20mm}%
		RO, Romania\\ \noindent\hspace*{20mm}%
		RS, Serbia\\ \noindent\hspace*{20mm}%
		SE, Sweden\\ \noindent\hspace*{20mm}%
		SI, Slovenia\\ \noindent\hspace*{20mm}%
		SK, Slovak Republic\\ \noindent\hspace*{20mm}%
		TK, Turkey\\ \noindent\hspace*{20mm}%
		UA, Ukraine\\ \noindent\hspace*{20mm}%
		UK, United Kingdom of Great Britain and Northern Ireland\\ \noindent\hspace*{10mm}%
	Source: European Value Survey 2008\\ \noindent\hspace*{10mm}%
	Treatment: Exclusion of Russian Federation and Azerbaijan. Merging of “Northern Ireland” and “Great Britain” \noindent\hspace*{10mm}%
	into “United Kingdom” and “Northern Cyprus” into “Cyprus”.\\
	
Region of residence: reg \\ \noindent\hspace*{10mm}%
	Variable class: factor \\ \noindent\hspace*{10mm}%
	Content: \\ \noindent\hspace*{20mm}%
		\\ \noindent\hspace*{20mm}%
		AL: Albania
		AT: Ostösterreich
		AT: Südösterreich
		AT: Westöstereich
		AM: Armenia
		BE: Région de Bruxelles-capitale/Brussels hoofdstedelijk gewest
		BE: Vlaams gewest
		BE: Région Wallonne
		BA: Bosna i Hercegovina
		BG: Severna i iztochna Bulgaria
		BG: Yugozapadna i yuzhna tsentralna Bulgaria
		BY: Belarus
		HR: Hrvatska
		CY: Kypros / Kibris
		CZ: Ceska Republika
		DK: Danmark
		EE: Eesti
		FI: Manner-Suomi
		FR: Île de France
		FR: Bassin Parisien
		FR: Nord-pas-de-Calais
		FR: Est
		FR: Ouest
		FR: Sud-Ouest
		FR: Centre-Est
		FR: Méditerranée
		GE: Georgia
		DE: Baden-Württemberg
		DE: Bayern
		DE: Berlin
		DE: Brandenburg
		DE: Bremen
		DE: Hamburg
		DE: Hessen
		DE: Mecklenburg-Vorpommern
		DE: Niedersachsen
		DE: Nordrhein-Westfalen
		DE: Rheinland-Pfalz
		DE: Saarland
		DE: Sachsen
		DE: Sachsen-Anhalt
		DE: Schleswig-Holstein
		DE: Thüringen
		GR: Voreia Ellada
		GR: Kentriki Ellada
		GR: Attiki
		GR: Nisia Aigaiou, Kriti
		HU: Közép-Magyarország
		HU: Dunántúl
		HU: Alföld és Észak
		IS: Ísland
		IE: Ireland
		IT: Nord-Ovest
		IT: Nord-Est
		IT: Centro (l)
		IT: Sud
		IT: Isole
		LV: Latvija
		LT: Lietuva
		LU: Luxembourg (Grand-Duché)
		MT: Malta
		MD: Moldova
		ME: Montenegro
		NL: Noord-Nederland
		NL: Oost-Nederland
		NL: West-Nederland
		NL: Zuid-Nederland
		NO: Norge
		PL: Region centralny
		PL: Region Poludinowy
		PL: Region Wschodni
		PL: Region Pólnocno-Zachodni
		PL: Region Póludniowo-Zachodni
		PL: Region Pólnocny
		PT: Continente
		PT: Região Autónoma dos Açores
		PT: Região Autónoma da Madeira
		RO: Macroregiunea unu
		RO: Macroregiunea doi
		RO: Macroregiunea trei
		RO: Macroregiunea patru
		RS: Centralna Srbija
		RS: Vojvodina
		SK: Slovenská Republika
		SI: Slovenija
		ES: Noroeste
		ES: Noreste
		ES: Comunidad de Madrid
		ES: Centro (E)
		ES: Este
		ES: Sur
		ES: Canarias
		SE: Östra Sverige
		SE: Södra Sverige
		SE: Norra Sverige
		CH: Schweiz/Suisse/Svizzera
		UA: West
		UA: Centre
		UA: North
		UA: East
		UA: South
		MK: Poranesnata jugoslovenska Republika Makedonija
		GB-GBN: North East (England)
		GB-GBN: North West (England)
		GB-GBN: Yorkshire and the Humber
		GB-GBN: East Midlands (England)
		GB-GBN: West Midlands (England)
		GB-GBN: East of England
		GB-GBN: London
		GB-GBN: South East (England)
		GB-GBN: South West (England)
		GB-GBN: Wales
		GB-GBN: Scotland
		GB-GBN: Northern Ireland
		RS-KM: Kosovo\\ 
		\noindent\hspace*{10mm}%
	Source: European Value Survey 2008\\ \noindent\hspace*{10mm}%
	Treatment: \\

 	Geographical region: eureg \\ \noindent\hspace*{10mm}%
 	Variable class: factor \\ \noindent\hspace*{10mm}%
 	Content: \\ 
 		\noindent\hspace*{20mm}%
 		Northern Europe (Denmark, Sweden)\\ \noindent\hspace*{20mm}%
 		Western Europe (Germany, Austria)\\ \noindent\hspace*{20mm}%
 		Southern Europe\\ \noindent\hspace*{20mm}%
 		Eastern Europe\\
 		
 	\noindent\hspace*{10mm}%	
 	Source: European Value Survey 2008\\ \noindent\hspace*{10mm}%
 	Treatment: Classification of regions is based on the UN methodology. Armenia has been added to Southern Europe and Georgia to Eastern Europe.\\
 	

\subsection{Work-related variables}
\subsection{Non-pecuniary happiness determinants}
\subsection{Pecuniary happiness determinants}
\subsection{Index on institutional trust}
\subsection{Index on norm salience}
\subsection{Demographic variables}
\subsection{Macro-variables}

%% The Appendices part is started with the command \appendix;
%% appendix sections are then done as normal sections
%% \appendix

%% \section{}
%% \label{}

%% If you have bibdatabase file and want bibtex to generate the
%% bibitems, please use
%%
%%  \bibliographystyle{elsarticle-harv} 
%%  \bibliography{<your bibdatabase>}

%% else use the following coding to input the bibitems directly in the
%% TeX file.

\begin{thebibliography}{00}

%% \bibitem[Author(year)]{label}
%% Text of bibliographic item

\bibitem[ ()]{}

\end{thebibliography}
\end{document}

\endinput
%%

